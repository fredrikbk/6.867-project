\documentclass[10pt]{article}

\usepackage{fullpage}
\usepackage{amsmath}
\usepackage{amsthm}
\usepackage{url}
\usepackage{relsize}
\usepackage{xspace}
\usepackage{subfigure}
\usepackage{graphicx,color}
\usepackage{amssymb}
\usepackage[margin=0.82in]{geometry}

\def\TODO#1{\noindent\textbf{[TODO: #1}]}

\begin{document}
\title{6.867 --- Project Proposal}
\author{Chris Johnson \and Fredrik Kjolstad}
\date{}
\maketitle

\section{Summary}
We propose to examine if the News, specifically the Wall Street Journal, can predict the performance of the Dow Jones Industrial Average.
This proposal is based on the project suggestion on the 6.867 project wiki and we will use the provided data for 2007.

We propose to implement three different classification techniques to predict the Dow performance: Naive Bayes, Support Vector Machines, and a third probabilistic technique. The third technique will be a probabilistic technique and will be either logistic regression or a technique from the second half of the course.
It will be decided later when we know more about the available options.

Our end goal is a bag predictor that can predict the Dow Jones performance with good accuracy.
In addition we will apply decision theory to the probabilistic technique to evaluate whether hedging is beneficial under several gain/loss models.

\section{Project Components}


\subsection{Feature Selection}
Feature selection for Naive Bayes is straight forward: each word is a feature.
However, for the SVM and probabilistic models it is not obvious.
Making each word a feature yields very high dimensionality with only two possible values in each dimension.

We will therefore explore alternative features as well as dimensionality reduction techniques to yield a lower dimension space.

The first idea we have is tagging words and making the feature vectors be counts of the tags. 
Tagging positive words such as ``bull'' and ``buy'' separately from negative words such as ``bear,'' ``crisis,'' and ``crash'' allows us to reduce the dimensionality of the space while keeping some semantic information.

A more complicated idea is to do some kind of latent semantic analysis~\footnote{http://en.wikipedia.org/wiki/Latent\_semantic\_analysis} with Singular Value Decomposition or Principle Components Analysis. We believe that this kind of pre-processing would be easy to implement in Matlab.

\subsection{Implementations}


\subsection{Experiments}

\subsection{Evaluation}


\section{Milestones}



\end{document}
